\documentclass[a4paper,10pt]{article}
\usepackage[T1]{fontenc}
%\usepackage{url}
\usepackage[fleqn]{amsmath}	%[fleqn] venstrestiller likningene
\usepackage[utf8]{inputenc}
\usepackage{verbatim}
\usepackage{amsthm, amssymb, bm, gensymb}
\usepackage[norsk]{babel}
\usepackage[]{siunitx}
\usepackage{array}% http://ctan.org/pkg/array
\usepackage{booktabs}% http://ctan.org/pkg/booktabs
\usepackage{float}
\usepackage{enumitem}
\usepackage{graphicx}
\usepackage{rotating}
\usepackage{tikz}
\usepackage{multirow, bigdelim}
\usepackage{listings}
\usepackage{filecontents}
\usepackage{xcolor}
\usepackage[toc,page]{appendix}
\usepackage{caption}
\usepackage{subcaption}
\usepackage{multicol}
\definecolor{mygreen}{RGB}{28,172,0} % color values Red, Green, Blue
\definecolor{mylilas}{RGB}{170,55,241}
\renewcommand{\thesection}{\arabic{section}}
\title{Compuslory assignment 1}
%\date{}
\author{Elisabeth Christensen}

\begin{document}
\maketitle
\subsection*{Problem 1}
\begin{enumerate}[label=\alph*)]
\item By creating a scatter plot of the numerical data with the logarithm of the number of cars as an independent covariant, we see clearly that the concentration of NO$_2$ increases as the amount of cars increases as depicted in figure~\ref{}

\begin{figure}
\centering
\includegraphics[width=7cm]{cars_NO2.pdf}
\end{figure}

\end{enumerate}
\end{document}




